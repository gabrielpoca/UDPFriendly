\documentclass{llncs}
\usepackage{times}
% \usepackage[T1]{fontenc}

% Comentar para not MAC Users
\usepackage[utf8]{inputenc}

\usepackage[portuges]{babel}

\usepackage{a4}
%\usepackage[margin=3cm,nohead]{geometry}
\usepackage{epstopdf}
\usepackage{graphicx}
\usepackage{fancyvrb}
\usepackage{amsmath}
%\renewcommand{\baselinestretch}{1.5}

% Para ter os objectos no sitio que se pretende com o H.
\usepackage{float}

% Indentar automaticamente
%\usepackage{indentfirst}


\begin{document}
\mainmatter
\title{UDP Friendly}

\titlerunning{UDP Friendly}

\author{Gabriel Poça \and Maria Alves \and Tiago Ribeiro}

\authorrunning{Gabriel Poça \and Maria Alves \and Tiago Ribeiro}

\institute{
University of Minho, Department of  Informatics, 4710-057 Braga, Portugal\\
e-mail: \{a56974,a0000,a00000\}@alunos.uminho.pt
}

\date{}
\bibliographystyle{splncs}

\maketitle
%\begin{abstract}
%\end{abstract}

% Introdução
\section{Introdução}
Este documento serve como

\subsection{Tipos de Mensagens}
\begin{description}
	\item[SYN] Mensagem para requesitar o establecer de uma ligação com o servidor.
	\item[SYN\_ACK] Mensagem de resposta ao SYN por parte do servidor.
	\item[CON\_ACK] Mensagem de resposta ao SYN_ACK por parte do cliente de modo a terminar o establecer da ligação.
\end{description}

%UNCOMMENT se necessário
%De acordo com o ilustrado na Figura~\ref{fig:controller}
%% Exemplo para inserção de uma figura
%\begin{figure}
%\begin{center}
%\includegraphics[scale=0.40]{figura.pdf} 
%\end{center}
%\caption{\label{fig:controller}Architecture of the unified QoS metric fuzzy controller.}
%\end{figure} 

%\section{Simulation Scenario}

\section{Conclusões}


%UNCOMMENT para a bibliografia 
%% ficheirodebibliografia.bib
%\bibliography{ficheirodebibliografia}

%ou inserir directamente os vários \bibitem 

\begin{thebibliography}{1}

%\bibitem{korea}
%Jeong Ho Kwaka, Bong Gyou Leeb:
%\newblock{Estimating demand curve in the Korean VoIP telecommunications market}

\end{thebibliography}

\end{document}
