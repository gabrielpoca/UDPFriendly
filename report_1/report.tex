\documentclass{llncs}
\usepackage{times}
% \usepackage[T1]{fontenc}

% Comentar para not MAC Users
\usepackage[utf8]{inputenc}

\usepackage[portuges]{babel}

\usepackage{a4}
%\usepackage[margin=3cm,nohead]{geometry}
\usepackage{epstopdf}
\usepackage{graphicx}
\usepackage{fancyvrb}
\usepackage{amsmath}
%\renewcommand{\baselinestretch}{1.5}

% Para ter os objectos no sitio que se pretende com o H.
\usepackage{float}

% Indentar automaticamente
%\usepackage{indentfirst}


\begin{document}
\mainmatter
\title{UDP Friendly}

\titlerunning{UDP Friendly}

\author{Gabriel Poça \and Maria Alves \and Tiago Ribeiro}

\authorrunning{Gabriel Poça \and Maria Alves \and Tiago Ribeiro}

\institute{
University of Minho, Department of  Informatics, 4710-057 Braga, Portugal\\
e-mail: \{a56974,a0000,a54752\}@alunos.uminho.pt
}

\date{}
\bibliographystyle{splncs}

\maketitle
%\begin{abstract}
%\end{abstract}

% Introdução
\section{Notas Iniciais}
Este documento serve como resumo intermédio de trabalho para o projecto prático de Comunicação por Computadores. O documento aborda as metodologias e conceitos adoptados.

\section{Comunicação}
\subsection{Tipos de Mensagens}
Cada mensagem tranpsorta diferentes elementos de informação, como uma mensagem INFO que transporta informação do documento a enviar e um indicador da sequência da mesma. Existem seis tipos de mensagens:
\begin{description}
	\item[SYN] Mensagem inicial para o establecer da comunicação com o servidor.
	\item[SYN\_ACK] Mensagem de confirmação de SYN.
	\item[INFO] Mensagem que transporta informação sobre o documento a enviar.
	\item[ACK] Mensagem de confirmação da recepção de uma mensagem INFO.
	\item[FIN] Mensagem de final de comunicação.
	\item[FIN\_ACK] Mensagem de confirmação da recepção de FIN.
\end{description}

\subsection{Protocolo}
\subsubsection{Establecer comunicação}
A comunicação toma inicio com a mensagem \textit{SYN} por parte do client. O servidor recebe a mensagem em determinada porta e envia a resposta, \textit{SYN\_ACK}, por outra. O cliente deve então continuar a comunicação da mesma. Após establecer da comunicação o cliente pode enviar o documento ou terminar a comunicação.
\subsubsection{Envio do Documento de Texto}
O upload do documento é realizado através de mensagens \textit{INFO}. Uma mensagem \textit{INFO} é constituida por informação (conjunto de bytes) e um indicador da posição da respectiva informação na sequencia que constitui o documento enviado.

\subsubsection{Finalizar comunicação}

%UNCOMMENT se necessário
%De acordo com o ilustrado na Figura~\ref{fig:controller}
%% Exemplo para inserção de uma figura
%\begin{figure}
%\begin{center}
%\includegraphics[scale=0.40]{figura.pdf} 
%\end{center}
%\caption{\label{fig:controller}Architecture of the unified QoS metric fuzzy controller.}
%\end{figure} 


\end{document}
