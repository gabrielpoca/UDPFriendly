\documentclass{llncs}
\usepackage{times}
% \usepackage[T1]{fontenc}

% Comentar para not MAC Users
\usepackage[utf8]{inputenc}

\usepackage[portuges]{babel}

\usepackage{a4}
%\usepackage[margin=3cm,nohead]{geometry}
\usepackage{epstopdf}
\usepackage{graphicx}
\usepackage{fancyvrb}
\usepackage{amsmath}
%\renewcommand{\baselinestretch}{1.5}

% Para ter os objectos no sitio que se pretende com o H.
\usepackage{float}

% Indentar automaticamente
%\usepackage{indentfirst}


\begin{document}
\mainmatter
\title{UDP Friendly}

\titlerunning{UDP Friendly}

\author{Gabriel Poça \and Maria Alves \and Tiago Ribeiro}

\authorrunning{Gabriel Poça \and Maria Alves \and Tiago Ribeiro}

\institute{
University of Minho, Department of  Informatics, 4710-057 Braga, Portugal\\
e-mail: \{a56974,a54807,a54752\}@alunos.uminho.pt
}

\date{}
\bibliographystyle{splncs}

\maketitle
%\begin{abstract}
%\end{abstract}

% Introdução
\section{Notas Iniciais}
Este documento serve como resumo, intermédio, para acompanhamento do projecto prático de Comunicação por Computadores. São abordadas as metodologias e os conceitos adoptados.

\section{Arquitectura}
A arquitectura 

\section{Comunicação}
\subsection{Tipos de Mensagens}
Cada mensagem tranpsorta diferentes elementos de informação, por exemplo, um mensagem INFO transporta informação do documento a enviar e um indicador de sequência. Como tal existem seis tipos de mensagens:
\begin{description}
	\item[SYN] Mensagem inicial no establecer da comunicação com o servidor.
	\item[SYN\_ACK] Mensagem de confirmação de SYN.
	\item[INFO] Mensagem que transporta informação sobre o documento a enviar.
	\item[ACK] Mensagem de confirmação da recepção de uma mensagem INFO.
	\item[FIN] Mensagem de final de comunicação.
	\item[FIN\_ACK] Mensagem de confirmação da recepção de FIN.
\end{description}
No processo de comunicação entre cliente e servidor o primeiro apenas comunica com mensagens \textit{SYN}, \textit{INFO} e \textit{FIN} e o servidor com as restantes.

\subsection{Protocolo}
\subsubsection{Handshake}
A comunicação tem inicio com o envio da mensagem \textit{SYN} pelo client. O servidor recebe a mensagem e responde com \textit{SYN\_ACK}. Através da mensagem \textit{SYN\_ACK} o cliente é informado da porta para a qual deverá continuar a comunicação. Establecida a comunicação o cliente pode enviar um documento e terminar a comunicação.

\subsubsection{Upload}
O \textit{upload} é realiza-se através de mensagens \textit{INFO}. Cada mensagem é constituida por informação (conjunto de bytes a enviar para o servidor) e um indicador da posição da mesma informação numa sequencia que constitui o documento.

\subsubsection{Terminar}
A comunicação termina quando o cliente envia a mensagem \textit{FIN} à qual o servidor deve responder com \textit{FIN\_ACK} procedendo então à descodificação da informação recebida.

\subsection{Timeout e Janela}

%UNCOMMENT se necessário
%De acordo com o ilustrado na Figura~\ref{fig:controller}
%% Exemplo para inserção de uma figura
%\begin{figure}
%\begin{center}
%\includegraphics[scale=0.40]{figura.pdf} 
%\end{center}
%\caption{\label{fig:controller}Architecture of the unified QoS metric fuzzy controller.}
%\end{figure} 


\end{document}
